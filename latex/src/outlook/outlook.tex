\chapter{Outlook}
In conclusion, edge computing platforms support the operation and management of edge systems. Edge computing platforms mitigate or completely remove problems which can occur on IoT systems like the increased latency, increased bandwidth or privacy concerns. In the future, it remains to be seen how the platforms will develop, as some of them are still in an early stage of development.

\bigskip
In order to be able to assess the edge platforms even better, it would be appropriate to subject them to a longer field test in a real project. Further with the records of long term field tests better statements and decisions can be made about the practicability of the individual edge computing platforms. Also further research on how to connect existing IoT devices and factory machines into edge computing platforms should be done.

\bigskip
In the future, it will be interesting to see how existing systems with a large IoT devices base migrate to edge computing platforms and what complications arise. Currently, the problem is that the platforms sometimes feel more or less like alpha or beta. But it remains to be seen whether this will not change in the future. It also remains to be seen which edge computing platform companies prefer to choose in order to migrate existing systems or even create new systems. Finally, there is not only the choice between the platforms discussed in this thesis, but there are many more platforms including commercial platforms that have not been mentioned here at all. It therefore remains to be seen how the industry in general will adopt edge computing platforms.
